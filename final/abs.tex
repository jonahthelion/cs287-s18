
\documentclass[11pt]{article}

\usepackage{common}
\usepackage{hyperref}
\title{Abstract}
\author{Jonah Philion \\ jonahphilion@college.harvard.edu \\ \href{https://github.com/jonahthelion/cs287-s18/}{github}}
\begin{document}

\maketitle{}
\begin{enumerate}
\item Area - Unsupervised NMT via universal intermediate.\\ Unsupervised NMT has been done with universal decoder and encoder, and with universal encoder but not decoder, but it hasn't been done with neither. By using a universal intermediate language, I hope to be able to use character-level embeddings and a single training step to do UNMT, both of which haven't been done as far as I know.

\item Papers - \begin{itemize}\item \href{https://arxiv.org/pdf/1710.11041.pdf}{UNMT, Artetxe}
 \item\href{https://arxiv.org/pdf/1711.00043.pdf}{UNMT Using Monolingual Corpora Only, Lample}
\item\href{https://arxiv.org/pdf/1611.04558.pdf}{Zero Shot, Johnson}
\item\href{https://arxiv.org/pdf/1704.06960.pdf}{Translating Neuralese, Andreas}
\end{itemize}
Unsupervised style transfer papers will also be helpful.\\
Zero Shot or Neuralese would probably be most interesting to present to the class.

\item Time - Signed up for April 19.

\item Baselines are the implementation of UNMT from Artetxe and Lample. Artetxe has published code \href{https://github.com/artetxem/undreamt}{here} which will be a good guide but Lample has not published yet.

\item Team - Jonah Philion (jonahphilion@college.harvard.edu)
\end{enumerate}

\end{document}
